\documentclass{article}
\usepackage[utf8]{inputenc}
\usepackage{graphicx}
\setlength{\parindent}{0pt}
\setlength{\parskip}{1em}

\title{FOAR705 - Proof of Concept Design}
\author{Jan Jugueta}
\date{September 2019}

\begin{document}

\maketitle

\section*{Introduction}

Building on the work in Elaboration II, this Proof of Concept will focus on the collation of newspaper articles, storing them as .txt files, analysing them using Voyant Tools and then organising my notes of the analysis in an efficient way. True to the ever evoling nature of my research, I have found a tool that I think will be helpful in my note taking. I did not mention this solution in Elaboration II because I had not discovered this tool then. The tool is called Open Semantic Desktop Search, which will be used alongside VirtualBox and Voyant Tools.

\textbf{Note:} This Proof of Concept will start at the point the sources have been copied from \textit{Neues Deutschland} and saved in a directory on my machine as .txt files. As this process is fairly straight forward (just need to copy the text from the website and paste it in a .txt file), this Proof of Concept will assume this work has already been done.

\section*{User Stories}

\subsection*{Identify Themes}

As a research student, I would like to use software that is able to analyse a corpus of text and identify emergent themes thereof.

\subsection*{Store Analysis}

As a research student, I would like to store the data captured from the textual analysis onto my machine.

\subsection*{Additional Notes}

As a research student, I would like to store any notes I have written about the analysis separate to the data captured so not to corrupt the original data.

\subsection*{Tag Newspaper Articles}

As a research student, I would like software that would allow me to tag themes or keywords to the newspaper articles that have been analysed.

\subsection*{Tag Analysis}

As a research student, I would like software that would allow me to tag themes or keywords to the data captured from the textual analysis.

\end{document}
