\documentclass{article}
\usepackage[utf8]{inputenc}
\usepackage{graphicx}
\setlength{\parindent}{0pt}
\setlength{\parskip}{1em}

\title{FOAR705 - Proof of Concept Design}
\author{Jan Jugueta}
\date{September 2019}

\begin{document}

\maketitle

\section*{Introduction}

Building on the work in Elaboration II, this Proof of Concept will focus on the collation of newspaper articles, storing them as .txt files, analysing them using Voyant Tools and then organising my notes of the analysis in an efficient way. True to the ever evoling nature of my research, I have found a tool that I think will be helpful in my note taking. I did not mention this solution in Elaboration II because I had not discovered this tool then. The tool is called Open Semantic Desktop Search, which will be used alongside VirtualBox and Voyant Tools. I had previously stated I would store notes with Microsoft OneNote, but considering its inability to back up, I have decided to use .txt files with Open Semantic Desktop Search.

\textbf{Note:} This Proof of Concept will start at the point the sources have been copied from \textit{Neues Deutschland} and saved in a directory on my machine as .txt files. As this process is fairly straight forward (just need to copy the text from the website and paste it in a .txt file), this Proof of Concept will assume this work has already been done.

\section*{User Stories}

\subsection*{Identify Themes}

As a research student, I would like to use software that is able to analyse a corpus of text and identify emergent themes thereof.

\subsection*{Store Analysis}

As a research student, I would like to store the data captured from the textual analysis onto my machine.

\subsection*{Additional Notes}

As a research student, I would like to store any notes I have written about the analysis separate to the data captured so not to corrupt the original data.

\subsection*{Create Tags}

As a research student, I would like to create a set of tags that will help me organise all my data (including articles, textual analysis data and research notes.

\subsection*{Search My Research}

As a research student, I would like software that would allow me to search through all newspaper articles, data from textual analysis and notes.

\subsection*{Tag Newspaper Articles}

As a research student, I would like software that would allow me to tag themes or keywords to the newspaper articles that have been analysed.

\subsection*{Tag Analysis}

As a research student, I would like software that would allow me to tag themes or keywords to the data captured from the textual analysis.

\subsection*{Tag Notes}

As a research student, I would like software that would allow me to tag themes or keywords to the notes that I have written as part of my research.

\section*{Acceptance Criteria}

\subsection*{Identify Themes}

As a research student, I should be able to:
\begin{enumerate}
    \item Go to www.voyant-tools.org
    \item Click on Upload
    \item Navigate to the directory that contains the newspaper articles in .txt format
    \item Select the .txt files I want to analyse
    \item Select Choose
\end{enumerate}

\subsection*{Store Analysis}

As a research student, I should be able to:
\begin{enumerate}
    \item Select the Export button in Voyant Tools
    \item Select Export Current Data
    \item Select export all available data as tab separated values (text)
    \item Copy the text that appears in the new browser window
    \item Open a blank .txt file
    \item Paste the data in tab separated values in the .txt file
    \item Save the .txt file in the /Jugueta/MRes/data/textualanalysis/ directory with an appropriate file name
\end{enumerate}

\subsection*{Additional Notes}

As a research student, I should be able to:
\begin{enumerate}
    \item Open a new .txt file
    \item Write any notes that I have about the analysed text in a .txt file
    \item Save the .txt file in the /Jugueta/MRes/notes/textualanalysis/ directory
\end{enumerate}

\subsection*{Create Tags}

As a research student, I should be able to:
\begin{enumerate}
    \item Launch VirtualBoxVM
    \item Set the Shared Folder to /Jugueta/MRes/
    \item Start Open Semantic Desktop Search
    \item Click on the Activities menu in Open Semantic Desktop Search
    \item Click on Index documents to ensure that I am working with the right directory
    \item Click on Manage structure
    \item Click on Add new entry
    \item Enter a name for the tag and select tag in Facet type
    \item Click Save
\end{enumerate}

\subsection*{Search My Research}

As a research student, I should be able to:
\begin{enumerate}
    \item Launch VirtualBoxVM
    \item Start Open Semantic Desktop Search
    \item Use the Search option to find items relevant to the search query
    \item Use the Advanced Search to be more succinct with my search
\end{enumerate}

\subsection*{Tag Newspaper Articles}

As a research student, I should be able to:
\begin{enumerate}
    \item Launch VirtualBoxVM
    \item Start Open Semantic Desktop Search
    \item Find a newspaper article using the Search feature in Open Semantic Desktop Search
    \item Click the Tagging \& annotation button
    \item Go to the Tags tab
    \item Click on the appropriate tag(s)
\end{enumerate}

\subsection*{Tag Analysis}

As a research student, I should be able to:
\begin{enumerate}
    \item Launch VirtualBoxVM
    \item Start Open Semantic Desktop Search
    \item Find data from textual analysis using the Search feature in Open Semantic Desktop Search
    \item Click the Tagging \& annotation button
    \item Go to the Tags tab
    \item Click on the appropriate tag(s)
\end{enumerate}

\subsection*{Tag Notes}

As a research student, I should be able to:
\begin{enumerate}
    \item Launch VirtualBoxVM
    \item Start Open Semantic Desktop Search
    \item Find the notes I have made for my research using the Search feature in Open Semantic Desktop Search
    \item Click the Tagging \& annotation button
    \item Go to the Tags tab
    \item Click on the appropriate tag(s)
\end{enumerate}

\section*{Prerequisites}

All stories require that the first three stories: \textbf{Identify Themes}, \textbf{Store Analysis} and \textbf{Additional Notes} are completed (ideally in order as presented) before continuing with the other stories.

The stories of \textbf{Create Tags} and \textbf{Search my Research} do not necessarily go in any particular order. To explain, one could use \textbf{Search my Research} to find emergent themes in the notes, data, articles themselves and then conceive of a tag that would be useful. Conversely, one could have tagged some items then use the search feature to find them again. 

The stories of \textbf{Tag Newspaper Articles}, \textbf{Tag Analysis}, \textbf{Tag Notes} will require that the items are found before they are tagged, so it would be safe to assume that \textbf{Search my Research} would need to happen before these three stories. However, once tags have been assigned to newspaper articles, analyses and notes, \textbf{Search My Research} could be used again after the tagging has been performed.

\end{document}
